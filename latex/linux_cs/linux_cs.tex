% Template is from https://wch.github.io/latexsheet/

\documentclass[10pt,landscape]{article}
\usepackage{multicol}
\usepackage{calc}
\usepackage{ifthen}
\usepackage[landscape]{geometry}
\usepackage{hyperref}

% To make this come out properly in landscape mode, do one of the following
% 1.
%  pdflatex latexsheet.tex
%
% 2.
%  latex latexsheet.tex
%  dvips -P pdf  -t landscape latexsheet.dvi
%  ps2pdf latexsheet.ps


% If you're reading this, be prepared for confusion.  Making this was
% a learning experience for me, and it shows.  Much of the placement
% was hacked in; if you make it better, let me know...


% 2008-04
% Changed page margin code to use the geometry package. Also added code for
% conditional page margins, depending on paper size. Thanks to Uwe Ziegenhagen
% for the suggestions.

% 2006-08
% Made changes based on suggestions from Gene Cooperman. <gene at ccs.neu.edu>


% To Do:
% \listoffigures \listoftables
% \setcounter{secnumdepth}{0}


% This sets page margins to .5 inch if using letter paper, and to 1cm
% if using A4 paper. (This probably isn't strictly necessary.)
% If using another size paper, use default 1cm margins.
\ifthenelse{\lengthtest { \paperwidth = 11in}}
	{ \geometry{top=.5in,left=.5in,right=.5in,bottom=.5in} }
	{\ifthenelse{ \lengthtest{ \paperwidth = 297mm}}
		{\geometry{top=1cm,left=1cm,right=1cm,bottom=1cm} }
		{\geometry{top=1cm,left=1cm,right=1cm,bottom=1cm} }
	}

% Turn off header and footer
\pagestyle{empty}
 

% Redefine section commands to use less space
\makeatletter
\renewcommand{\section}{\@startsection{section}{1}{0mm}%
                                {-1ex plus -.5ex minus -.2ex}%
                                {0.5ex plus .2ex}%x
                                {\normalfont\large\bfseries}}
\renewcommand{\subsection}{\@startsection{subsection}{2}{0mm}%
                                {-1explus -.5ex minus -.2ex}%
                                {0.5ex plus .2ex}%
                                {\normalfont\normalsize\bfseries}}
\renewcommand{\subsubsection}{\@startsection{subsubsection}{3}{0mm}%
                                {-1ex plus -.5ex minus -.2ex}%
                                {1ex plus .2ex}%
                                {\normalfont\small\bfseries}}
\makeatother

% Define BibTeX command
\def\BibTeX{{\rm B\kern-.05em{\sc i\kern-.025em b}\kern-.08em
    T\kern-.1667em\lower.7ex\hbox{E}\kern-.125emX}}

% Don't print section numbers
\setcounter{secnumdepth}{0}


\setlength{\parindent}{0pt}
\setlength{\parskip}{0pt plus 0.5ex}


% -----------------------------------------------------------------------

\begin{document}

\raggedright
\footnotesize
\begin{multicols}{2} %{3}


% multicol parameters
% These lengths are set only within the two main columns
%\setlength{\columnseprule}{0.25pt}
\setlength{\premulticols}{1pt}
\setlength{\postmulticols}{1pt}
\setlength{\multicolsep}{1pt}
\setlength{\columnsep}{2pt}

\begin{center}
     %\Large{\textbf{\LaTeXe\ Cheat Sheet}} \\
     \Large{\textbf{Linux Command Line} Cheat Sheet} \\
\end{center}

\section{System Information}
\subsection{General}

%\texttt{} &  \\
\begin{tabular}{@{}ll@{}}
\texttt{uname -a}    & Linux system information \\
\texttt{cat /etc/redhat-release}  & Linux distribution version  \\
\texttt{hostname} & System host name \\
\texttt{hostname -I}  & System IP address  \\
\texttt{ifconfig -a}  & Display network interfaces and ip address \\
\texttt{date}  & Current date and time \\
\texttt{w}  & Display who is logged into the server  \\
\texttt{whoami} & Who are you logged in as  
\end{tabular}

%Used at the very beginning of a document:
%\verb!\documentclass{!\textit{class}\verb!}!.  Use
%\verb!\begin{document}! to start contents and \verb!\end{document}! to
%end the document.


\subsection{Hardware}
\begin{tabular}{@{}ll@{}}
\texttt{free -h} & Display free and used RAM/Memory \\
\texttt{df -h} & Display free and used space in file system \\
\texttt{fdisk -l} & Display the disk partitions size and types \\
\texttt{du -ah} & Display the disk usage for all files and directories
\end{tabular}

\section{File and Directory Commands}
\subsection{Navigation}
\begin{tabular}{@{}ll@{}}
\texttt{pwd} & Display the current working directory \\
\texttt{cd ..} & Go up one level \\
\texttt{cd } & Go to your home directory \\
\texttt{cd ~/Downloads} & Go to the Downloads directory inside your home directory \\
\texttt{cd /dev/null} & Navigate to the /dev/null directory 
\end{tabular}

\subsection{Files}
\begin{tabular}{@{}ll@{}}
\texttt{ls -al} & Display all files in detail \\
\texttt{rm file\_name} & Remove/delete a file \\
\texttt{rm -r directory\_name} & Recursively remove a directory and its contents \\
\texttt{cp file1 file2} & Copy file1 to file2  \\
\texttt{cp -r source\_dir destination} & Copy source recursively to destination \\
\texttt{mv file1 file2} & Move file1 to file2 \\
\texttt{ln -s /path/to/file linkname} & Create a symbolic link to linkname \\
\texttt{touch file\_name} & Creates and empty file or updates the access/modification info  \\
\texttt{cat file} & See the contents of a file \\
\texttt{less file} & Scroll through the file \\
\texttt{head file} & Display the first 10 lines \\
\texttt{tail file} & Display the last 10 lines \\
\texttt{tail -f file} & -f follows the file as it is appended too 
\end{tabular}

\section{File Permissions}
\begin{tabular}{@{}ll@{}}
\texttt{} & \\
\end{tabular}

\section{Searching}
\begin{tabular}{@{}ll@{}}
\texttt{} & \\
\end{tabular}

\section{Archiving Files}
\begin{tabular}{@{}ll@{}}
\texttt{} & \\
\end{tabular}

\section{Process Management}
\begin{tabular}{@{}ll@{}}
\texttt{} & \\
\end{tabular}

\section{SSH}
\begin{tabular}{@{}ll@{}}
\texttt{} & \\
\end{tabular}

\section{Transfering Files}
\begin{tabular}{@{}ll@{}}
\texttt{} & \\
\end{tabular}


%Usage:
%\verb!\documentclass[!\textit{opt,opt}\verb!]{!\textit{class}\verb!}!.

%---------------------------------------------------------------------------


%\section{Sample \LaTeX\ document}
%\begin{verbatim}
%\documentclass[11pt]{article}
% ...
%\end{verbatim}

%\rule{0.3\linewidth}{0.25pt}
%\scriptsize
%Copyright \copyright\ 2014 Winston Chang
%\href{http://wch.github.io/latexsheet/}{http://wch.github.io/latexsheet/}


\end{multicols}
\end{document}
